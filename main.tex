\section*{Explicit Two-Region Unified Model for the RBMK Reactor}

This system of \textbf{16 coupled Delay Differential Equations (DDEs)} explicitly separates the Lower ($L$) and Upper ($U$) core regions. This separation is required to model the \textbf{coolant transport delay} (which causes oscillations) and the \textbf{asymmetric SCRAM effect} (where the bottom of the core surges while the top shuts down).

\subsection*{1. State Vector Definitions}

The system state $\mathbf{y}(t)$ consists of 16 variables:
\[
\mathbf{y}(t) = 
\begin{bmatrix}
\mathbf{y}_L \\
\mathbf{y}_U
\end{bmatrix}
\]

For each region $i \in \{L, U\}$:
\begin{enumerate}
    \item $n_i$: Neutron Density (Normalized Power)
    \item $C_i$: Delayed Neutron Precursors
    \item $\alpha_i$: Void Fraction
    \item $T_{f,i}$: Fuel Temperature (K)
    \item $T_{m,i}$: Moderator Temperature (K)
    \item $I_i$: Iodine-135 Concentration
    \item $X_i$: Xenon-135 Concentration
    \item $c_i$: Control Rod Insertion Fraction ($0 \to 1$)
\end{enumerate}

\subsection*{2. Lower Core Equations (Trigger Region)}
This region receives cold-water inlet and the positive scram-tip effect first.

\[
\frac{dn_L}{dt} = \frac{\rho_L(t) - \beta}{\Lambda} n_L + \lambda_d C_L 
+ \frac{D_n}{H^2}(n_U - n_L)
\]

\[
\frac{dC_L}{dt} = \frac{\beta}{\Lambda} n_L - \lambda_d C_L
\]

\[
\frac{d\alpha_L}{dt} = 
\left[
\frac{\alpha_{eq}(n_L) - \alpha_L}{\tau_{void}}
\right]
\frac{1}{1 + k_{sat}\alpha_L}
\]

\[
\frac{dT_{f,L}}{dt} = a_f n_L - b_f (T_{f,L} - T_{coolant})
\]

\[
\frac{dT_{m,L}}{dt} = a_m n_L - b_m (T_{m,L} - T_{coolant})
\]

\[
\frac{dI_L}{dt} = y_I n_L - \lambda_I I_L
\]

\[
\frac{dX_L}{dt} = 
y_X n_L + \lambda_I I_L - (\lambda_X + \sigma_X n_L) X_L
\]

\[
\frac{dc_L}{dt} = \frac{c_{target} - c_L}{\tau_{rod}}
\]

\subsection*{3. Upper Core Equations (Follower Region)}
This region receives steam from below after a flow delay $\tau_{flow}$.

\[
\frac{dn_U}{dt} = 
\frac{\rho_U(t) - \beta}{\Lambda} n_U + \lambda_d C_U 
+ \frac{D_n}{H^2}(n_L - n_U)
\]

\[
\frac{dC_U}{dt} = \frac{\beta}{\Lambda} n_U - \lambda_d C_U
\]

\[
\frac{d\alpha_U}{dt} = 
\left[
\frac{\alpha_{eq}(n_U) - \alpha_U}{\tau_{void}}
+
\frac{2}{\tau_{flow}}
\bigl( \alpha_L(t - \tau_{flow}) - \alpha_U \bigr)
\right]
\frac{1}{1 + k_{sat}\alpha_U}
\]

\[
\frac{dT_{f,U}}{dt} = a_f n_U - b_f (T_{f,U} - T_{coolant})
\]

\[
\frac{dT_{m,U}}{dt} = a_m n_U - b_m (T_{m,U} - T_{coolant})
\]

\[
\frac{dI_U}{dt} = y_I n_U - \lambda_I I_U
\]

\[
\frac{dX_U}{dt} = 
y_X n_U + \lambda_I I_U - (\lambda_X + \sigma_X n_U) X_U
\]

\[
\frac{dc_U}{dt} = \frac{c_{target} - c_U}{\tau_{rod}}
\]

\subsection*{4. Auxiliary Constitutive Relations}

\subsubsection*{A. Unified High/Low Power Boiling Curve}

\[
x_i(n_i) = \frac{K_{heat} \, n_i}{m_{flow}}
\]

\[
\alpha_{eq}(n_i) = 
\alpha_{max}
\frac{x_i(n_i)^{0.25}}{1 + x_i(n_i)^{0.25}}
\]

\subsubsection*{B. Lower Core Reactivity}

\[
\rho_L(t) = 
\kappa_V \alpha_L
+ \kappa_D(\sqrt{T_{f,L}} - \sqrt{T_0})
- \kappa_X X_L
+ \rho_{rod,L}(c_L)
\]

Rod function with positive SCRAM-tip spike:
\[
\rho_{rod,L}(c_L) = 
\kappa_{tip} \, c_L \, e^{-10 c_L}
- \kappa_{boron} \, c_L
\]

\subsubsection*{C. Upper Core Reactivity}

\[
\rho_U(t) = 
\kappa_V \alpha_U
+ \kappa_D(\sqrt{T_{f,U}} - \sqrt{T_0})
- \kappa_X X_U
+ \rho_{rod,U}(c_U)
\]

\[
\rho_{rod,U}(c_U) = -\kappa_{boron} c_U
\]

\subsection*{5. Key Parameters}

\begin{table}[h!]
\centering
\begin{tabular}{lll}
\toprule
Symbol & Description & Physical Role \\
\midrule
$\alpha_L(t - \tau_{flow})$ & Delayed State & Drives oscillations; couples regions \\
$0.25$ & Shape Parameter & Sharp flash-boiling behavior \\
$\sigma_X n X$ & Burnout & Creates the Xenon pit at low power \\
$\kappa_{tip}$ & Tip Worth & Positive spike from control-rod tips \\
$D_n$ & Neutron Coupling & Links lower/upper neutron fields \\
\bottomrule
\end{tabular}
\end{table}

\section*{Unified Model Regime Behavior}

The entire purpose of the ``Unified'' model design is that \textbf{one set of equations} naturally produces the correct behavior in both regimes without needing to manually switch equations or use ``if statements'' to change the physics.

Here is exactly how the 16-equation model automatically adapts to each regime:

\subsection*{1. High Power Regime (100\% Power)}

\textbf{Behavior:} Stable, load-following.

\medskip
\noindent\textbf{Why it works:}
\begin{itemize}
    \item \textbf{The Boiling Curve:} At high power ($n \approx 1.0$), the steam quality $x$ is high. The logistic function
    \[
        \frac{x^{0.25}}{1 + x^{0.25}}
    \]
    flattens out. This means a change in power produces a \emph{linear} and predictable change in void fraction.
    \item \textbf{The Doppler Effect:} At high power, the fuel temperature $T_f$ is high (on the order of $\sim 800~\mathrm{K}$). The term $\sqrt{T_f}$ is large, providing strong \textbf{negative feedback}.
\end{itemize}

\noindent\textbf{Result:} If power bumps up, the strong Doppler effect immediately pushes it back down. The reactor is stable.

\subsection*{2. Low Power Regime ($< 20\%$ Power)}

\textbf{Behavior:} Unstable, ``hair-trigger'', oscillatory.

\medskip
\noindent\textbf{Why it works:}
\begin{itemize}
    \item \textbf{The Boiling Curve:} At low power ($n \approx 0.05$), steam quality $x$ is near zero. On the curve $x^{0.25}$, the slope is nearly vertical. A tiny change in power causes a \textbf{massive} spike in void fraction (``flash boiling'').
    \item \textbf{The Doppler Effect:} At low power, the fuel temperature is low (on the order of $\sim 200~\mathrm{K}$). The term $\sqrt{T_f}$ is small and weak. It cannot overcome the massive positive void feedback.
\end{itemize}

\noindent\textbf{Result:} The negative feedback (brakes) is gone, and the positive feedback (gas pedal) is floored. The reactor becomes unstable.

\subsection*{3. The Transition (The ``Xenon Pit'')}

\textbf{Behavior:} The reactor gets ``stuck'' and requires rod withdrawal.

\medskip
\noindent\textbf{Why it works:}
\begin{itemize}
    \item Because you included the \textbf{Iodine ODE} ($dI/dt$), when you simulate the ramp down from high to low power, the iodine does not disappear. It sits there decaying, pushing up the xenon concentration $X$.
    \item This forces the controller (or the operator) to withdraw the rods ($c \to 0$) to keep the reactor alive, setting the trap for the accident.
\end{itemize}

\subsection*{Implementation ``Safety'' Tip (Crucial for MATLAB)}

While the equations are valid, there is one numerical danger in the low-power regime.

The term $n^{0.25}$ (or $x^{0.25}$) has an \textbf{infinite derivative} at exactly zero. If the MATLAB solver takes a step where $n$ becomes slightly negative (e.g.\ $-1 \times 10^{-9}$) due to numerical noise, $n^{0.25}$ will become a complex number ($a + bi$), and the simulation will crash or produce non-physical results.

\medskip
\noindent\textbf{The Fix:} When implementing the boiling curve in your function, protect the root calculation:
\begin{verbatim}
% In your unified_rbmk_ode function:

% Prevent negative power/quality from breaking the root function
n_safe = max(n_L, 1e-6); % Clamp to a tiny positive number
x_L = (K_heat * n_safe) / flow;

% Now calculate alpha safely
alpha_eq = alpha_max * (x_L^0.25) / (1 + x_L^0.25);
\end{verbatim}

\medskip
\noindent\textbf{Conclusion:} Use the 16-equation model. It is the only way to correctly simulate the \emph{journey} from a stable reactor to an exploding one.